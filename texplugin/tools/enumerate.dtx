% \iffalse meta-comment
%
% Copyright 1993-2016
%
% The LaTeX3 Project and any individual authors listed elsewhere
% in this file.
%
% This file is part of the Standard LaTeX `Tools Bundle'.
% -------------------------------------------------------
%
% It may be distributed and/or modified under the
% conditions of the LaTeX Project Public License, either version 1.3c
% of this license or (at your option) any later version.
% The latest version of this license is in
%    http://www.latex-project.org/lppl.txt
% and version 1.3c or later is part of all distributions of LaTeX
% version 2005/12/01 or later.
%
% The list of all files belonging to the LaTeX `Tools Bundle' is
% given in the file `manifest.txt'.
%
% \fi
% \iffalse
%% File: enumerate.dtx Copyright  1993 1994 1999 David Carlisle
%
%<*dtx>
          \ProvidesFile{enumerate.dtx}
%</dtx>
%<package>\NeedsTeXFormat{LaTeX2e}
%<package>\ProvidesPackage{enumerate}
%<driver> \ProvidesFile{enumerate.drv}
% \fi
%         \ProvidesFile{enumerate.dtx}
          [2015/07/23 v3.00 enumerate extensions (DPC)]
%
% \iffalse
%<*driver>
\documentclass{ltxdoc}
\usepackage{enumerate}
\begin{document}
\DocInput{enumerate.dtx}
\end{document}
%</driver>
% \fi
%
% \GetFileInfo{enumerate.dtx}
% \title{The \textsf{enumerate} package\thanks{This file
%        has version number \fileversion, last
%        revised \filedate.}}
% \author{David Carlisle}
% \date{\filedate}
% \MaintainedByLaTeXTeam{tools}
% \maketitle
%
% %%%%%%%%%%%%%%%%%%%%%%%%%%%%%%%%%%%%%%%%%%%%%%%%%%%%%%%%%%%%%%%%%%%%
%
%
% \changes{v1.00}{1990/00/00}{First version (since lost)}
% \changes{v2.00}{1991/07/25}{New Algorithm}
% \changes{v2.01}{1992/06/26}{Re-issue for the new doc and docstrip.}
% \changes{v2.02}{1994/01/31}{Update for LaTeX2e}
% \changes{v3.00}{1999/03/05}{Add extension hook for tools/2916}
% \changes{v3.00}{2015/07/23}{Documentation fixes tools/2916}
%
% \begin{abstract}
% This package gives the enumerate environment an optional argument
% which determines the style in which the counter is printed.
%
% An occurrence of one of the tokens |A a I i| or |1| produces the value
% of the counter printed with (respectively) |\Alph \alph \Roman \roman|
% or |\arabic|.
%
% These letters may be surrounded by any strings involving any other
% \TeX\ expressions, however the tokens |A a I i 1| must be inside a
% |{ }| group if they are not to be taken as special.
% \end{abstract}
%
% \section{Examples}
%\showboxdepth=10
% \showboxbreadth=10
% \[
% \parbox{2.2in}{\hbadness2000
% \begin{enumerate}[EX i.]
% \item one one one one one one one
%        one one one one one\label{LA}
% \item two
%     \begin{enumerate}[{example} a)]
%     \item one of two  one of two  one of two\label{LB}
%     \item two of two
%     \end{enumerate}
% \end{enumerate}
% \begin{enumerate}[{A}-1]
% \item one\label{LC}
% \item two
% \end{enumerate}}
% \hspace{20pt}
% \vcenter{\hsize=2.4in
% \begin{verbatim}
% \begin{enumerate}[EX i.]
% \item one one one one one one one
%        one one one one\label{LA}
% \item two
%     \begin{enumerate}[{example} a)]
%     \item one of two  one of two
%           one of two\label{LB}
%     \item two of two
%     \end{enumerate}
% \end{enumerate}
%
% \begin{enumerate}[{A}-1]
% \item one\label{LC}
% \item two
% \end{enumerate}
% \end{verbatim}
% }\]
%
% |\label| and |\ref| may be used as with the standard {\tt
% enumerate} environment. |\ref| only produces the counter value,
% not the whole label. |\ref| prints the value in the same style
% as |\item|, as determined by the presence of one of the tokens
% |A a I i 1| in the optional argument. In the above example
% |\ref{LA}|, |\ref{LB}| and |\ref{LC}| produce
% `\ref{LA}',  `\ref{LB}' and `\ref{LC}' respectively.
%
% \StopEventually{}
%
%\section{Macros}
%
%    \begin{macrocode}
%<*package>
%    \end{macrocode}
%
% \begin{macro}{\@enlab}
% Internal token register used to build up the label command from the
% optional argument.
%    \begin{macrocode}
\newtoks\@enLab
%    \end{macrocode}
% \end{macro}
%
% \begin{macro}{\@enQmark}
% This just expands to a `?'. |\ref| will produce this, if no counter
% is printed.
%    \begin{macrocode}
\def\@enQmark{?}
%    \end{macrocode}
% \end{macro}
%
% The next four macros build up the command that will print the item
% label. They each gobble one token or group from the optional argument,
% and add corresponding tokens to the register |\@enLab|. They each end
% with a call to |\@enloop|, which starts the processing of the next
% token.
% \begin{macro}{\@enLabel}
% Add the counter to the label. |#2| will be one of the `special'
% tokens |A a I i 1|, and is thrown away. |#1| will be a command
% like |\Roman|.
%    \begin{macrocode}
\def\@enLabel#1#2{%
  \edef\@enThe{\noexpand#1{\@enumctr}}%
  \@enLab\expandafter{\the\@enLab\csname the\@enumctr\endcsname}%
  \@enloop}
%    \end{macrocode}
% \end{macro}
%
% \begin{macro}{\@enSpace}
% \begin{macro}{\@enSp@ce}
% Add a space to the label. The tricky bit is to gobble the space token,
% as you can not do this with a macro argument.
%    \begin{macrocode}
\def\@enSpace{\afterassignment\@enSp@ce\let\@tempa= }
\def\@enSp@ce{\@enLab\expandafter{\the\@enLab\space}\@enloop}
%    \end{macrocode}
% \end{macro}
% \end{macro}
%
% \begin{macro}{\@enGroup}
% Add a |{ }| group to the label.
%    \begin{macrocode}
\def\@enGroup#1{\@enLab\expandafter{\the\@enLab{#1}}\@enloop}
%    \end{macrocode}
% \end{macro}
%
% \begin{macro}{\@enOther}
% Add anything else to the label
%    \begin{macrocode}
\def\@enOther#1{\@enLab\expandafter{\the\@enLab#1}\@enloop}
%    \end{macrocode}
% \end{macro}
%
% \begin{macro}{\@enloop}
% \begin{macro}{\@enloop@}
% The body of the main loop.
% Eating tokens this way instead of using |\@tfor| lets you see
% spaces and {\bf all} braces. |\@tfor| would treat {\tt a} and
% |{a}| as  special, but not |{{a}}|.
%    \begin{macrocode}
\def\@enloop{\futurelet\@entemp\@enloop@}
%    \end{macrocode}
%    \begin{macrocode}
\def\@enloop@{%
  \ifx A\@entemp         \def\@tempa{\@enLabel\Alph  }\else
  \ifx a\@entemp         \def\@tempa{\@enLabel\alph  }\else
  \ifx i\@entemp         \def\@tempa{\@enLabel\roman }\else
  \ifx I\@entemp         \def\@tempa{\@enLabel\Roman }\else
  \ifx 1\@entemp         \def\@tempa{\@enLabel\arabic}\else
  \ifx \@sptoken\@entemp \let\@tempa\@enSpace         \else
  \ifx \bgroup\@entemp   \let\@tempa\@enGroup         \else
  \ifx \@enum@\@entemp   \let\@tempa\@gobble          \else
                         \let\@tempa\@enOther
%    \end{macrocode}
%    Hook for possible extensions
%    \begin{macrocode}
                         \@enhook
%    \end{macrocode}
%
%    \begin{macrocode}
             \fi\fi\fi\fi\fi\fi\fi\fi
%    \end{macrocode}
% \end{macro}
% Process the current token, then look at the next.
%    \begin{macrocode}
  \@tempa}
%    \end{macrocode}
% \end{macro}
%
% \begin{macro}{\@enhook}
% Hook for possible extensions.
% Some packages may want to extend the number of special characters
% that are associated with counter representations. This feature
% was requested to enable Russian alphabetic counting, but here
% I give an example of a footnote symbol counter, triggered by |*|.
%
% To enable a new counter type based on a letter, you just need
% to add a new |\ifx| clause by analogy with the code above.
% So for example to make |*| trigger footnote symbol counting.
% a package should do the following.
%
% Initialise the hook, in case the package is loaded before
% \textsf{enumerate}.
%\begin{verbatim}
% \providecommand\@enhook{}
%\end{verbatim}
%
% Add to the hook a new |\ifx| clause that associates |*| with the
% |\fnsymbol| counter command.
%\begin{verbatim}
% \g@addto@macro\@enhook{%
%   \ifx *\@entemp
%     \def\@tempa{\@enLabel\fnsymbol}%
%   \fi}
%\end{verbatim}
% This code sequence should work whether it is loaded before or after
% this enumerate package. Any number of new counter types may be added
% in this way.
%
% At this point we just need initialise the hook, taking care not
% to over write any definitions another package may already have added.
%    \begin{macrocode}
\providecommand\@enhook{}
%    \end{macrocode}
% \end{macro}
%
% \begin{macro}{\enumerate}
% The new {\tt enumerate} environment. This is the first half of the
% original enumerate environment. If there is an optional argument, call
% |\@@enum@| to define the label commands, otherwise call
% |\@enum@| which is the second half of the original definition.
%    \begin{macrocode}
\def\enumerate{%
  \ifnum \@enumdepth >3 \@toodeep\else
      \advance\@enumdepth \@ne
      \edef\@enumctr{enum\romannumeral\the\@enumdepth}\fi
  \@ifnextchar[{\@@enum@}{\@enum@}}
%    \end{macrocode}
% \end{macro}
%
% \begin{macro}{\@@enum@}
% Handle the optional argument..
%    \begin{macrocode}
\def\@@enum@[#1]{%
%    \end{macrocode}
% Initialise the loop which will break apart the optional argument.
% The command to print the label is built up in |\@enlab|.
% |\@enThe| will be used to define |\theenum|\,$n$.
%    \begin{macrocode}
  \@enLab{}\let\@enThe\@enQmark
%    \end{macrocode}
% The |\@enum@| below is never expanded, it is used to detect the end
% of the token list.
%    \begin{macrocode}
  \@enloop#1\@enum@
%    \end{macrocode}
% Issue a warning if we did not find one of the `special' tokens.
%    \begin{macrocode}
  \ifx\@enThe\@enQmark\@warning{The counter will not be printed.%
   ^^J\space\@spaces\@spaces\@spaces The label is: \the\@enLab}\fi
%    \end{macrocode}
% Define |\labelenum|$\,n$ and |\theenum|$\,n$.
%    \begin{macrocode}
  \expandafter\edef\csname label\@enumctr\endcsname{\the\@enLab}%
  \expandafter\let\csname the\@enumctr\endcsname\@enThe
%    \end{macrocode}
% Set the counter to 7 so that we get the width of `vii' if roman
% numbering is in force then set |\leftmargin|$\,n$. to the width of the
% label plus |\labelsep|.
%    \begin{macrocode}
  \csname c@\@enumctr\endcsname7
  \expandafter\settowidth
            \csname leftmargin\romannumeral\@enumdepth\endcsname
            {\the\@enLab\hspace{\labelsep}}%
%    \end{macrocode}
% Finally call |\@enum@| which is the second half of the original
% definition.
%    \begin{macrocode}
  \@enum@}
%    \end{macrocode}
% \end{macro}
%
% \begin{macro}{\@enum@}
% All the list parameters have now been defined, so call |\list|. This
% is taken straight from the original definition of |\enumerate|.
%    \begin{macrocode}
\def\@enum@{\list{\csname label\@enumctr\endcsname}%
           {\usecounter{\@enumctr}\def\makelabel##1{\hss\llap{##1}}}}
%    \end{macrocode}
% \end{macro}
%
%    \begin{macrocode}
%</package>
%    \end{macrocode}
%
% \Finale
%
\endinput
